\documentclass[conference]{IEEEtran}
\IEEEoverridecommandlockouts
% The preceding line is only needed to identify funding in the first footnote. If that is unneeded, please comment it out.
\usepackage{cite}
\usepackage{amsmath,amssymb,amsfonts}
\usepackage{algorithmic}
\usepackage{graphicx}
\usepackage{textcomp}
\usepackage{xcolor}
\def\BibTeX{{\rm B\kern-.05em{\sc i\kern-.025em b}\kern-.08em
    T\kern-.1667em\lower.7ex\hbox{E}\kern-.125emX}}
\begin{document}

\title{Virtual Neurorobotics in the Human Brain Project}

\author{\IEEEauthorblockN{1\textsuperscript{st} Given Name Surname}
\IEEEauthorblockA{\textit{dept. name of organization (of Aff.)} \\
\textit{name of organization (of Aff.)}\\
City, Country \\
email address}
\and
\IEEEauthorblockN{2\textsuperscript{nd} Given Name Surname}
\IEEEauthorblockA{\textit{dept. name of organization (of Aff.)} \\
\textit{name of organization (of Aff.)}\\
City, Country \\
email address}
\and
\IEEEauthorblockN{3\textsuperscript{rd} Given Name Surname}
\IEEEauthorblockA{\textit{dept. name of organization (of Aff.)} \\
\textit{name of organization (of Aff.)}\\
City, Country \\
email address}
\and
}

\maketitle

\begin{abstract}

\end{abstract}

\begin{IEEEkeywords}
Spiking Neural Networks, Evolutionary Algorithms, Neurorobotics Platform, Human Brain Project
\end{IEEEkeywords}

\section{Introduction}


\section{Evolutionary Algorithms}


\section{Spiking Neural Networks}


\section{Hardcoded Approaches}

\subsection{Inverse Kinematics for Grasping}

\subsection{Hardcoded Throwing}

\subsubsection{State Machine}


\section{Learned Approaches}

\subsection{Evolutionary Approach}

\subsection{Simplified Problem}


\section{Results}


\section{Problems}

\section{Conclusion}





\section*{Acknowledgment}



\section*{References}


\begin{thebibliography}{00}
\bibitem{b1} G. Eason, B. Noble, and I. N. Sneddon, ``On certain integrals of Lipschitz-Hankel type involving products of Bessel functions,'' Phil. Trans. Roy. Soc. London, vol. A247, pp. 529--551, April 1955.
\end{thebibliography}


\end{document}
